\documentclass[a4paper,notitlepage,10pt]{article}

\usepackage[T1]{fontenc} 
\usepackage[utf8]{inputenc} 
\usepackage[italian]{babel}
\usepackage{microtype}

\usepackage[a4paper,top=3cm,bottom=3cm,left=2.5cm,right=2.5cm,%
heightrounded,bindingoffset=5mm]{geometry}

\usepackage{amsmath,amssymb}
\usepackage{siunitx}
\sisetup{
separate-uncertainty,
separate-uncertainty-units = bracket}

\usepackage{graphicx}	% provides \includegraphics e \resizebox
% \graphicspath{{./immagini_J}}
\usepackage{caption}
% \captionsetup{tableposition=top, figureposition=bottom, font=small, format=hang, labelfont=bf} 
\usepackage{subfig}		% requires caption

\usepackage{floatrow}
% \floatsetup[table]{style=plaintop} % altrim mette le caption dle table in basso (nonostante il setup di caption)
\newfloatcommand{capbtabbox}{table}[][\FBwidth] % Table float box with bottom caption, box width adjusted to content

\usepackage{booktabs}



\begin{document}
%\numberwithin{equation}{section} %Per numerare equazioni 

% due righe sullo scopo dell'esperienza/misura

\section{Strumentazione e componenti}

\section[Procedura]{Procedura e discussione della misura}

\section{Osservazioni}





\end{document}


% Per inserire una MISURA
\SI{12.4(13)e2}{\milli\volt} dà $(12.4\pm1.3) \times 10^{2} mV$



% Per affiancare DUE FIGURE, si può mettere una caption globale
\begin{figure}[htbp]
\centering
	\subfloat[][Example specific caption %\label{fig:example_label}
	]{
	\resizebox{.45\textwidth}{!}{
   \includegraphics{file_name} % nome dell'immagine
%   \includegraphics{Jacopo/immagini_J/file_name} % per overleaf
	}}\qquad
	\subfloat[][Example specific caption %\label{fig:example_label}
	]{
	\resizebox{.45\textwidth}{!}{
   \includegraphics{file_name} % nome dell'immagine
%   \includegraphics{Jacopo/immagini_J/file_name} % per overleaf
	}}
	\caption{Example global caption} 
%	\label{fig:example_global_label}
\end{figure}




% Per affiancare DUE FIGURE con caption distinte
\begin{figure}[htbp]
\begin{floatrow}
   \centering
   \ffigbox{%
\resizebox{.45\textwidth}{!}{ 
   \includegraphics{file_name} % da modificare
%   \includegraphics{Jacopo/immagini_J/file_name} % per overleaf
   }}{%
   \caption{Example caption}
%.  \label{fig:example_label}
}\quad
\ffigbox{%
\resizebox{.45\textwidth}{!}{
   \includegraphics{file_name} % da modificare
%   \includegraphics{Jacopo/immagini_J/file_name} % per overleaf
   }}{%
   \caption{Example caption}
%   \label{fig:example_label}
   }
\end{floatrow}
\end{figure}




% Per le TABELLE
\begin{table}[htbp]
  \begin{tabular}{
	S[table-format = 1.2(2)]  
	S[table-format = 2.1(1)e1]} 
 % Qua si sta specificando il formato dei numeri che andremo a inserire in ciascuna colonna della tabella. A destra del punto è indicato il massimo numero di cifre dopo la virgola di cui abbiamo bisogno, in parentesi il numero di cifre dell'errore. Prima del punto invece è segnato il massimo numero di cifre della parte intera di quelli che inseriremo. Nella seconda colonna  e1 serve a dirgli che avremo bisogno della notazione esponenziale
			
 \toprule
{$f$ $[\unit{\hertz}]$} & {$v_{\textup{s}}$} \\ % le graffe sono necessarie!
	
 \midrule
1.23(4)	 & 13.8(4)e5 \\
% Questo è uguale a:  1.23 \pm 0.04 & (13.8 \pm 0.4) \times 10^5 \\
    
 \bottomrule
\end{tabular}
\caption{Example caption}
\label{tab:example_label}
\end{table}




% Ultimo consiglio spassionato: lasciamo [htbp], non forziamolo a mettere figure e tabelle dove vogliamo, ci pensiamo alla fine se non ci piace quello che ci sputa fuori sul documento finale